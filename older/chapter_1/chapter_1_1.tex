\section{Espacio vectorial normado}

\begin{definition*}
    Sea $X$ un espacio vectorial sobre
    $\mathbb{R}$ $\left(\mathbb{C}\right)$,
    $\left\|\phantom{\cdot}\right\|\colon X\to\mathbb{R}$
    se llama norma real (compleja) en $X$ si:
    \begin{enumerate}[i)]
        \item

              \begin{math}
                  \left\|x\right\|=
                  0\implies
                  x=
                  0
              \end{math}.
              Definida, Separación

        \item

              \begin{math}
                  \left\|\lambda x\right\|=
                  \left|\lambda\right|
                  \left\|x\right\|
                  \forall x\in X, \lambda\in\mathbb{R}
                  \left(\mathbb{C}\right)
              \end{math}.
              Absolutamente Homogénea,
              Homogeneidad Positiva

        \item

              \begin{math}
                  \left\|x+y\right\|\leq
                  \left\|x\right\|+
                  \left\|y\right\|
                  \forall x, y\in X
              \end{math}.
              Desigualdad Triangular, Subaditiva.
    \end{enumerate}

    $\left\|x\right\|$ se llama norma de $x$.
    $\left(X, \left\|\phantom{\cdot}\right\|\right)$ se
    llama espacio vectorial real (complejo) normado.
\end{definition*}

\begin{proposition*}
    Sea $X$ un espacio vectorial normado.
    Luego,
    \begin{math}
        x=
        0\implies
        \left\|x\right\|=
        0
    \end{math}
\end{proposition*}

\begin{proof}
    \begin{align*}
        x                & =0                               \\
        \left\|x\right\| & =\left\|0\right\|                \\
                         & =\left\|0x\right\|               \\
                         & = \left|0\right|\left\|x\right\| \\
                         & = 0\left\|x\right\|              \\
                         & =0
    \end{align*}
\end{proof}

\begin{proposition*}
    Sea $X$ un espacio vectorial normado.
    Luego, $x\neq0\implies\left\|x\right\|\neq0$
\end{proposition*}

\begin{proof}
    Asumamos que $\left\|x\right\|=0\implies x=0$.
    Contradicción.
    $\therefore\left\|x\right\|\neq 0$.
\end{proof}

\begin{proposition*}
    Sea $\left(X, \left\|\phantom{\cdot}\right\|\right)$
    un espacio vectorial normado.
    Luego: $\left\|x\right\|\geq0\,\forall x\in X$.
    Positiva
\end{proposition*}

\begin{proof}
    Se tiene que
    \begin{align*}
        \left\|x+y\right\|               & \leq
        \left\|x\right\|+
        \left\|y\right\|
        \shortintertext{Sea $y=-x$.}
        \left\|x+\left(-x\right)\right\| & \leq
        \left\|x\right\|+\left\|-x\right\|                        \\
        \left\|0\right\|                 & \leq 2\left\|x\right\| \\
        0                                & \leq2\left\|x\right\|
    \end{align*}
    \begin{math}
        \therefore
        \forall x\in X:
        \left\|x\right\|\geq
        0
    \end{math}.
\end{proof}

\begin{proposition*}
    Sea $\left\|\phantom{\cdot}\right\|$ una norma en $X$.
    Luego, $d\left(x,y\right)=\left\|x-y\right\|$ es una
    métrica en $X$.
    Es decir, todo espacio vectorial normado es un espacio
    métrico.
    $d$ se llama métrica inducida por la norma
    $\left\|\phantom{\cdot}\right\|$.
\end{proposition*}

\begin{proof}\leavevmode
    \begin{enumerate}[i)]
        \item $d\left(x,y\right)=0$ si y solo si $x=y$.


              \begin{math}
                  d\left(x,y\right)=
                  0\iff
                  \left\|x-y\right\|=
                  0\iff
                  x-y=
                  0\iff
                  x=
                  y
              \end{math}.

        \item

              \begin{math}
                  d\left(x,y\right)=
                  d\left(y,x\right)\forall
                  x,y\in X
              \end{math}.

              \begin{math}
                  d\left(x,y\right)=
                  \left\|x-y\right\|=
                  \left\|-\left(y-x\right)\right\|=
                  \left|-1\right|\left\|y-x\right\|=
                  \left\|y-x\right\|=
                  d\left(y,x\right).
              \end{math}


        \item

              \begin{math}
                  d\left(x,y\right)\leq
                  d\left(x,z\right)+d\left(z,y\right)\forall
                  x,y,z\in X
              \end{math}.

              \begin{math}
                  d\left(x,y\right)=
                  \left\|x-y\right\|=
                  \left\|x-z+z-y\right\|\leq
                  \left\|x-z\right\|+\left\|z-y\right\|=
                  d\left(x,z\right)+d\left(z,y\right)
              \end{math}.
    \end{enumerate}
\end{proof}

\begin{proposition*}
    Sea $\left(X, \left\|\phantom{\cdot}\right\|\right)$ un
    espacio vectorial normado.
    Luego:
    \begin{math}
        \left\|x\right\|\geq
        0\,\forall
        x\in X
    \end{math}.
\end{proposition*}

\begin{proof}
    \begin{math}
        d\left(x,y\right)=
        \left\|x-y\right\|
    \end{math}
    es una métrica
    en $X$.
    Se tiene que
    \begin{math}
        d\left(x,y\right)\geq
        0\,\forall
        x,y\in X\implies
        \left\|x-y\right\|\geq
        0\,\forall x,y\in X
    \end{math}.
    Sea
    \begin{math}
        y=
        0\implies
        \left\|x\right\|\geq
        0\,\forall
        x\in X
    \end{math}.
\end{proof}

\begin{proposition*}
    Sean $X$ un espacio vectorial,
    \begin{math}
        \left\{
        x_{1},x_{2},\dotsc,x_{n}
        \right\}\subset
        X
    \end{math}
    linealmente independiente.
    Si
    \begin{math}
        \left\|\phantom{\cdot}\right\|\colon
        \left\langle
        \left\{
        x_{1},x_{2},\dotsc,x_{n}
        \right\}
        \right\rangle\to
        \mathbb{R}
    \end{math}
    donde
    \begin{math}
        \left\|x\right\|=
        \sum_{k=1}^{n}
        \left|
        \lambda_{k}
        \right|
    \end{math}
    y
    \begin{math}
        x=
        \sum_{k=1}^{n}
        \lambda_{k}x_{k}
    \end{math},
    entonces
    \begin{math}
        \left\|
        x
        \right\|\geq
        0
    \end{math}.
\end{proposition*}

\begin{proof}
    \begin{math}
        \left|
        \lambda_{k}
        \right|\geq
        0\,\forall
        k\in\mathbb{N}\implies
        \sum_{k=1}^{n}
        \left|
        \lambda_{k}
        \right|\geq
        0\implies
        \left\|
        x
        \right\|\geq
        0
    \end{math}.
\end{proof}

\begin{proposition*}
    Sea $X$ un espacio vectorial normado.
    Luego, $x\neq 0$ si y solo si $\left\|x\right\|>0$.
\end{proposition*}

\begin{proof}\leavevmode
    \begin{itemize}
        \item[$\left(\Rightarrow\right)$]

              Se tiene que $\left\|x\right\|\geq 0\,\forall x\in X$.
              Asumamos que $\left\|x\right\|=0\implies x=0$.
              Contradicción.
              $\therefore \left\|x\right\|>0$.

        \item[$\left(\Leftarrow\right)$]

              Asumamos que
              \begin{math}
                  x=
                  0\implies
                  \left\|x\right\|=
                  0
              \end{math}.
              Contradicción
              $\therefore x\neq0$.
    \end{itemize}
\end{proof}

\begin{proposition*}
    Sea $\left(X, \left\|\phantom{\cdot}\right\|\right)$
    un espacio vectorial normado $n$-dimensional,
    \begin{math}
        \varphi\colon
        \mathbb{R}^{n}\to X
    \end{math}
    donde
    \begin{math}
        {\left\|x\right\|}_{\ast}=
        \left\|\varphi\left(x\right)\right\|
    \end{math}.
    Luego,
    \begin{math}
        {\left\|x\right\|}_{\ast}\geq
        0
    \end{math}.
\end{proposition*}

\begin{proof}
    \begin{align*}
        \shortintertext{Sea $x\in\mathbb{R}^{n}$.}
        \varphi\left(x\right)\in X
        \left\|\varphi\left(x\right)\right\|\geq
        0 \\
        {\left\|x\right\|}_{\ast}\geq
        0.
    \end{align*}
\end{proof}

\begin{proposition*}
    Sean $\left(X, \left\|\phantom{\cdot}\right\|\right)$
    un espacio vectorial normado de dimensión $n$,
    \begin{math}
        \left\{
        v_{1},\dotsc,
        v_{n}
        \right\}
    \end{math}
    una base de $X$,
    \begin{math}
        f\colon X\to\mathbb{R}^{n},
    \end{math}
    donde
    \begin{math}
        f\left(x\right)=
        \left(y_{1},\dotsc,y_{n}\right)
    \end{math},
    \begin{math}
        x=
        \sum_{k=1}^{n}
        y_{k}v_{k}
    \end{math}.
    Si
    \begin{math}
        {\left\|y\right\|}_{\ast}=
        \left\|f^{-1}\left(y\right)\right\|
    \end{math},
    entonces
    \begin{math}
        {\left\|y\right\|}_{\ast}\geq
        0
    \end{math}.
\end{proposition*}

\begin{proof}
    .
\end{proof}