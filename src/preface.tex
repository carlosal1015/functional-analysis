%%%%%%%%%%%%%%%%%%%%%%preface.tex%%%%%%%%%%%%%%%%%%%%%%%%%%%%%%%%%%%%%%%%%
% sample preface
%
% Use this file as a template for your own input.
%
%%%%%%%%%%%%%%%%%%%%%%%% Springer %%%%%%%%%%%%%%%%%%%%%%%%%%

\preface

Un prefacio\index{prefacio} es la declaración preliminar de un libro,
generalmente escrita por el \textit{autor o editor} de una obra, que
establece su origen, alcance, propósito, plan y público objetivo, y que
a veces incluye reflexiones posteriores y agradecimientos de asistencia.

Cuando está escrito por una persona distinta al autor, se le llama prólogo.
El prefacio o prólogo se diferencia de la introducción, que trata del
tema de la obra.

Habitualmente se incluyen \textit{agradecimientos} como última parte
del prefacio.

\vspace{\baselineskip}
\begin{flushright}\noindent
    Lima, marzo 2024\hfill {\it Héctor  Guimaray}
\end{flushright}