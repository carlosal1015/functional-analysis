\section{Espacio vectorial normado}

\begin{definition*}
    Sea $X$ un espacio vectorial sobre $\mathbb{R}$ $\left(\mathbb{C}\right)$, $\left\|\phantom{\cdot}\right\|\colon X\to\mathbb{R}$ se llama norma real (compleja) en $X$ si:
    \begin{enumerate}[i)]
        \item

              \begin{math}
                  \left\|x\right\|=
                  0\implies
                  x=0
              \end{math}.
              Definida, Separación

        \item

              \begin{math}
                  \left\|\lambda x\right\|=
                  \left|\lambda\right|
                  \left\|x\right\|
                  \forall x\in X, \lambda\in\mathbb{R} \left(\mathbb{C}\right)
              \end{math}.
              Absolutamente Homogénea,
              Homogeneidad Positiva

        \item

              \begin{math}
                  \left\|x+y\right\|\leq
                  \left\|x\right\|+
                  \left\|y\right\|
                  \forall x, y\in X
              \end{math}.
              Desigualdad Triangular, Subaditiva.
    \end{enumerate}
    $\left\|x\right\|$ se llama norma de $x$.
    $\left(X, \left\|\phantom{\cdot}\right\|\right)$ se llama espacio vectorial real (complejo) normado.
\end{definition*}

\begin{proposition*}
    Sea $X$ un espacio vectorial normado.
    Luego,
    \begin{math}
        x=0\implies
        \left\|x\right\|=0
    \end{math}
\end{proposition*}

\begin{proof}
    \begin{align*}
        x                & =0                               \\
        \left\|x\right\| & =\left\|0\right\|                \\
                         & =\left\|0x\right\|               \\
                         & = \left|0\right|\left\|x\right\| \\
                         & = 0\left\|x\right\|              \\
                         & =0
    \end{align*}
\end{proof}

\begin{proposition*}
    Sea $X$ un espacio vectorial normado.
    Luego, $x\neq0\implies\left\|x\right\|\neq0$
\end{proposition*}

\begin{proof}
    Asumamos que $\left\|x\right\|=0\implies x=0$.
    Contradicción.
    $\therefore\left\|x\right\|\neq 0$.
\end{proof}

\begin{proposition*}
    Sea $\left(X, \left\|\phantom{\cdot}\right\|\right)$ un espacio vectorial normado. Luego: $\left\|x\right\|\geq0\,\forall x\in X$. Positiva
\end{proposition*}

\begin{proof}
    Se tiene que:
    \begin{math}
        \left\|x+y\right\|\leq
        \left\|x\right\|+
        \left\|y\right\|
    \end{math}.
    Sea $y=-x$.
    \begin{align*}
        \left\|x+\left(-x\right)\right\| & \leq
        \left\|x\right\|+\left\|-x\right\|                        \\
        \left\|0\right\|                 & \leq 2\left\|x\right\| \\
        0                                & \leq2\left\|x\right\|  \\
        \left\|x\right\|                 & \geq0\,\forall x\in X
    \end{align*}
\end{proof}

\begin{proposition*}
    Sea $\left\|\phantom{\cdot}\right\|$ una norma en $X$.
    Luego, $d\left(x,y\right)=\left\|x-y\right\|$ es una métrica en $X$.
    Es decir, todo espacio vectorial normado es un espacio métrico $d$ se llama métrica inducida por la norma $\left\|\phantom{\cdot}\right\|$.
\end{proposition*}

\begin{proof}\leavevmode
    \begin{itemize}
        \item $d\left(x,y\right)=0$ si y solo si $x=y$.

              $d\left(x,y\right)=0\iff \left\|x-y\right\|=0\iff x-y=0\iff x=y$.

        \item $d\left(x,y\right)=d\left(y,x\right)\forall x,y\in X$.

              $d\left(x,y\right)=\left\|x-y\right\|=\left\|-\left(y-x\right)\right\|=\left|-1\right|\left\|y-x\right\|=\left\|y-x\right\|=d\left(y,x\right)$.

        \item $d\left(x,y\right)\leq d\left(x,z\right)+d\left(z,y\right)\forall x,y,z\in X$

              $d\left(x,y\right)=\left\|x-y\right\|=\left\|x-z+z-y\right\|\leq \left\|x-z\right\|+\left\|z-y\right\|=d\left(x,z\right)+d\left(z,y\right)$.
    \end{itemize}
\end{proof}

\begin{proposition}
    Sea $\left(X, \left\|\phantom{\cdot}\right\|\right)$ un espacio vectorial normado. Luego: $\left\|x\right\|\geq 0\forall x\in X$.
\end{proposition}

\begin{proof}
    $d\left(x,y\right)=\left\|x-y\right\|$ es una métrica en $X$.
    Se tiene que $d\left(x,y\right)\geq0\forall x,y\in X\implies \left\|x-y\right\|\geq 0\forall x,y\in X$.
    Sea $y=0\implies\left\|x\right\|\geq 0\forall x\in X$.
\end{proof}

\begin{proposition*}
    Sean $X$ un espacio vectorial, $\left\{x_{1},x_{2},\dotsc,x_{n}\right\}\subset X$ linealmente independiente. Si $\left\|\phantom{\cdot}\right\|\colon\left\langle x_{1},x_{2},\dotsc,x_{n}\right\rangle\to\mathbb{R}$ donde $\left\|x\right\|$.
    %∑𝑛𝑘=1|𝜆𝑘 | , y x = ∑𝑛𝑘=1 𝜆𝑘 𝑥𝑘 entonces ‖𝑥‖ ≥ 0
\end{proposition*}