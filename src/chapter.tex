%%%%%%%%%%%%%%%%%%%%% chapter.tex %%%%%%%%%%%%%%%%%%%%%%%%%%%%%%%%%
%
% sample chapter
%
% Use this file as a template for your own input.
%
%%%%%%%%%%%%%%%%%%%%%%%% Springer-Verlag %%%%%%%%%%%%%%%%%%%%%%%%%%
%\motto{Use the template \emph{chapter.tex} to style the various elements of your chapter content.}
\chapter{Espacio vectorial normado}
\label{intro} % Always give a unique label
% use \chaptermark{}
% to alter or adjust the chapter heading in the running head
\abstract{Cada capítulo debe ir precedido de un resumen (no más de 200 palabras) que resuma el contenido.}

\begin{definition}
    Sea $X$ un espacio vectorial sobre
    \begin{math}
        \mathbb{R}
        \left(
        \mathbb{C}
        \right)
    \end{math},
    \begin{math}
        \left\|
        \phantom{\cdot}
        \right\|\colon
        X\to
        \mathbb{R}
    \end{math}
    se llama norma real (compleja) en $X$ si:
    \begin{enumerate}[i)]
        \item

              \begin{math}
                  \left\|
                  x
                  \right\|=
                  0\implies
                  x=
                  0
              \end{math}.
              \hfill
              Definida, Separación

        \item

              \begin{math}
                  \left\|
                  \lambda x
                  \right\|=
                  \left|
                  \lambda
                  \right|
                  \left\|
                  x
                  \right\|
                  \forall x\in X,
                  \lambda\in\mathbb{R}
                  \left(\mathbb{C}\right)
              \end{math}.
              \hfill
              Absolutamente Homogénea,

              \hfill
              Homogeneidad Positiva

        \item

              \begin{math}
                  \left\|
                  x+y
                  \right\|\leq
                  \left\|
                  x
                  \right\|+
                  \left\|
                  y
                  \right\|
                  \forall x, y\in X
              \end{math}.
              \hfill
              Desigualdad Triangular, Subaditiva.
    \end{enumerate}

    \begin{math}
        \left\|
        x
        \right\|
    \end{math}
    se llama norma de $x$.
    \begin{math}
        \left(
        X,
        \left\|
        \phantom{\cdot}
        \right\|
        \right)
    \end{math}
    se llama espacio vectorial real (complejo) normado.
\end{definition}

\section{Subespacio vectorial normado}

% \clearpage
\begin{prob}
    \label{prob1}
    Sea $X$ un espacio vectorial normado.
    Luego,
    \begin{math}
        x=
        0\implies
        \left\|
        x
        \right\|=
        0
    \end{math}.
\end{prob}

\begin{prob}
    \label{prob2}
    Sea $X$ un espacio vectorial normado.
    Luego,
    \begin{math}
        x\neq
        0\implies
        \left\|
        x
        \right\|\neq
        0
    \end{math}.
\end{prob}

\begin{prob}
    \label{prob3}
    Sea
    \begin{math}
        \left(
        X,
        \left\|
        \phantom{\cdot}
        \right\|
        \right)
    \end{math}
    un espacio vectorial normado.
    Luego:
    \begin{math}
        \left\|
        x
        \right\|\geq
        0\,
        \forall x\in X
    \end{math}.
    \hfill
    Positiva
\end{prob}

\begin{prob}
    \label{prob4}
    Sea
    \begin{math}
        \left\|
        \phantom{\cdot}
        \right\|
    \end{math}
    una norma en $X$.
    Luego,
    \begin{math}
        d\left(x,y\right)=
        \left\|
        x-y
        \right\|
    \end{math}
    es una métrica en $X$.
    Es decir, todo espacio vectorial normado es un espacio
    métrico.
    $d$ se llama métrica inducida por la norma
    \begin{math}
        \left\|
        \phantom{\cdot}
        \right\|
    \end{math}.
\end{prob}


\begin{prob}
    \label{prob5}
    Sea
    \begin{math}
        \left(
        X,
        \left\|
        \phantom{\cdot}
        \right\|
        \right)
    \end{math}
    un espacio vectorial normado.
    Luego:
    \begin{math}
        \left\|
        x
        \right\|\geq
        0\,\forall
        x\in X
    \end{math}.
\end{prob}

\begin{prob}
    \label{prob6}
    Sean $X$ un espacio vectorial,
    \begin{math}
        \left\{
        x_{1},
        x_{2},
        \dotsc,
        x_{n}
        \right\}\subset
        X
    \end{math}
    linealmente independiente.
    Si
    \begin{math}
        \left\|
        \phantom{\cdot}
        \right\|\colon
        \left\langle
        \left\{
        x_{1},
        x_{2},
        \dotsc,
        x_{n}
        \right\}
        \right\rangle\to
        \mathbb{R}
    \end{math}
    donde
    \begin{math}
        \left\|
        x
        \right\|=
        \sum_{k=1}^{n}
        \left|
        \lambda_{k}
        \right|
    \end{math}
    y
    \begin{math}
        x=
        \sum_{k=1}^{n}
        \lambda_{k}x_{k}
    \end{math},
    entonces
    \begin{math}
        \left\|
        x
        \right\|\geq
        0
    \end{math}.
\end{prob}

\begin{prob}
    \label{prob7}
    Sea $X$ un espacio vectorial normado.
    Luego, $x\neq 0$ si y solo si $\left\|x\right\|>0$.
\end{prob}

\begin{prob}
    \label{prob8}
    Sea
    \begin{math}
        \left(
        X,
        \left\|
        \phantom{\cdot}
        \right\|
        \right)
    \end{math}
    un espacio vectorial normado $n$-dimensional,
    \begin{math}
        \varphi\colon
        \mathbb{R}^{n}\to
        X
    \end{math}
    donde
    \begin{math}
        {\left\|x\right\|}_{\ast}=
        \left\|
        \varphi
        \left(
        x
        \right)
        \right\|
    \end{math}.
    Luego,
    \begin{math}
        {\left\|x\right\|}_{\ast}\geq
        0
    \end{math}.
\end{prob}

\begin{prob}
    \label{prob9}
    Sean
    \begin{math}
        \left(
        X,
        \left\|
        \phantom{\cdot}
        \right\|
        \right)
    \end{math}
    un espacio vectorial normado de dimensión $n$,
    \begin{math}
        \left\{
        v_{1},\dotsc,
        v_{n}
        \right\}
    \end{math}
    una base de $X$,
    \begin{math}
        f\colon X\to\mathbb{R}^{n},
    \end{math}
    donde
    \begin{math}
        f\left(x\right)=
        \left(
        y_{1},
        \dotsc,
        y_{n}
        \right)
    \end{math},
    \begin{math}
        x=
        \sum_{k=1}^{n}
        y_{k}v_{k}
    \end{math}.
    Si
    \begin{math}
        {\left\|y\right\|}_{\ast}=
        \left\|
        f^{-1}
        \left(y\right)
        \right\|
    \end{math},
    entonces
    \begin{math}
        {\left\|y\right\|}_{\ast}\geq
        0
    \end{math}.
\end{prob}

\begin{prob}
    \label{prob10}
    Sean $X$ un espacio vectorial,
    \begin{math}
        \left\{
        x_{1},
        x_{2},
        \dotsc,
        x_{n}
        \right\}\subset X
    \end{math}
    linealmente independiente.
    Luego,
    \begin{math}
        \left\|
        \phantom{\cdot}
        \right\|\colon
        \left\langle
        \left\{
        x_{1},
        x_{2},
        \dotsc,
        x_{n}
        \right\}
        \right\rangle\to
        \mathbb{R}
    \end{math}
    donde
    \begin{math}
        \left\|
        x
        \right\|=
        \sum_{k=1}^{n}
        \left|
        \lambda_{k}
        \right|
    \end{math}
    y
    \begin{math}
        x=
        \sum_{k=1}^{n}
        \lambda_{k}x_{k}
    \end{math},
    es una norma.
\end{prob}

% BibTeX users please use
% \bibliographystyle{}
% \bibliography{}
%
%\biblstarthook{

\begin{thebibliography}{99.}%
	% and use \bibitem to create references.
	%
	% Use the following syntax and markup for your references if
	% the subject of your book is from the field
	% "Mathematics, Physics, Statistics, Computer Science"
	%
	% Contribution
	\bibitem{science-contrib} Broy, M.: Software engineering --- from auxiliary to key technologies. In: Broy, M., Dener, E. (eds.) Software Pioneers, pp. 10-13. Springer, Heidelberg (2002)
	%
	% Online Document
	\bibitem{science-online} Dod, J.: Effective substances. In: The Dictionary of Substances and Their Effects. Royal Society of Chemistry (1999) Available via DIALOG. \\
	\url{http://www.rsc.org/dose/title of subordinate document. Cited 15 Jan 1999}
	%
	% Monograph
	\bibitem{science-mono} Geddes, K.O., Czapor, S.R., Labahn, G.: Algorithms for Computer Algebra. Kluwer, Boston (1992)
	%
	% Journal article
	\bibitem{science-journal} Hamburger, C.: Quasimonotonicity, regularity and duality for nonlinear systems of partial differential equations. Ann. Mat. Pura. Appl. \textbf{169}, 321--354 (1995)
	%
\end{thebibliography}


\section{Isomorfismo en espacio vectorial normado}

\section{Distancia en espacio vectorial normado}

\section{Diámetro de conjunto en espacio vectorial normado}

\section{Bola en espacio vectorial normado}

\section{Esfera en espacio vectorial normado}

\section{Normas equivalentes en espacio vectorial normado}

\section{Conjunto abierto en espacio vectorial normado}

\section{Conjunto cerrado en espacio vectorial normado}

\section{Vecindad en espacio vectorial normado}

\section{Entorno en espacio vectorial normado}

\section{Topología fuerte en espacio vectorial normado}

\section{Punto interior en espacio vectorial normado}

\section{Punto exterior en espacio vectorial normado}

\section{Punto de frontera en espacio vectorial normado}

\section{Punto aislado en espacio vectorial normado}

\section{Punto de acumulación en espacio vectorial normado}

\section{Conjunto Bolzano-Weierstrass en espacio vectorial normado}

\section{Punto de clausura en espacio vectorial normado}

\section{Conjunto separado en espacio vectorial normado}

\section{Conjunto denso en espacio vectorial normado}

\section{Espacio vectorial normado separable}

\section{Lema de Riesz en espacio vectorial normado}

\section{Sucesión en espacio vectorial normado}

\section{Punto límite de sucesión en espacio vectorial normado}

\section{Límite de sucesión en espacio vectorial normado}

\section{Sucesión de Cauchy en espacio vectorial normado}

\section{Teorema de Bolzano-Weierstrass para conjunto infinito en espacio vectorial normado finito dimensional}

\section{Teorema de Bolzano-Weierstrass para sucesión en espacio vectorial normado finito dimensional}

\section{Base de Schauder en espacio vectorial normado}

\section{Base de Hilbert en espacio vectorial normado}

\section{Función en espacio vectorial normado}

\section{Isometría en espacio vectorial normado}

\section{Función contraída en espacio vectorial normado}

\section{Límite de función en espacio vectorial normado}

\section{Función continua en espacio vectorial normado}

\section{Función uniformemente continua en espacio vectorial normado}

\section{Función homeomorfa en espacio vectorial normado}

\section{Sucesión de funciones en espacio vectorial normado}

\section{Límite de sucesión de funciones en espacio vectorial normado}

\section{Serie de funciones en espacio vectorial normado}

\section{Conjunto acotado en espacio vectorial normado}

\section{Conjunto totalmente acotado en espacio vectorial normado}

\section{Conjunto completo en espacio vectorial normado}

\section{Función lineal continua en espacio vectorial normado}

\section{Función lineal acotada en espacio vectorial normado}

\section{Espacio de funciones lineales acotadas en espacio vectorial normado}

\section{Colección puntualmente acotado en espacio vectorial normado}

\section{Colección uniformemente acotado en espacio vectorial normado}

\section{Espacio dual topológico en espacio vectorial normado}

\section{Conjunto compacto en espacio vectorial normado}

\section{Conjunto conexo en espacio vectorial normado}

\section{Conjunto convexo en espacio vectorial normado}

\chapter{Espacio de Banach}

\section{Espacio de Banach}

\chapter{Espacio producto interno}

\section{Espacio producto interno}

\section{Ortogonalidad en espacio producto interno}

\section{Teorema de Pitágoras en espacio producto interno}

\section{Desigualdad de Bessel en espacio producto interno}

\chapter{Espacio de Hilbert}

\section{Espacio de Hilbert}

\section{Límite de sucesión espacio de Hilbert}

\section{Teorema de Banach-Steinhaus en espacio de Hilbert}

\section{Teorema de Representación de Riesz en espacio de Hilbert}

\section{Límite débil de sucesión en espacio de Hilbert}

\section{Teorema de descomposición ortogonal en espacio de Hilbert}

\chapter{Teoremas importantes}

\section{Teorema de extensión de Hahn-Banach en espacio vectorial}

\section{Principio de Acotación Uniforme en espacio de Banach}

\section{Teorema del gráfico cerrado en espacio de Banach}

\section{Teorema de la función abierta en espacio de Banach}

% \begin{theorem}
%     Theorem text goes here.
% \end{theorem}

% \begin{proof}
%     %\smartqed
%     Proof text goes here.
%     %\qed
% \end{proof}