\section*{Problemas del Capítulo~\ref{intro}}

\begin{sol}{prob1}
	\begin{align*}
		x                & =0                              \\
		\left\|x\right\| & =\left\|0\right\|               \\
		                 & =\left\|0x\right\|              \\
		                 & =\left|0\right|\left\|x\right\| \\
		                 & =0\left\|x\right\|              \\
		                 & =0
	\end{align*}
\end{sol}

\begin{sol}{prob2}
	Asumamos que $\left\|x\right\|=0\implies x=0$.
	Contradicción.
	$\therefore\left\|x\right\|\neq 0$.
\end{sol}

\begin{sol}{prob3}
	Se tiene que:
	\begin{align*}
		\left\|x+y\right\|               & \leq
		\left\|x\right\|+
		\left\|y\right\|
		\shortintertext{Sea $y=-x$.}
		\left\|x+\left(-x\right)\right\| & \leq
		\left\|x\right\|+\left\|-x\right\|                        \\
		\left\|0\right\|                 & \leq 2\left\|x\right\| \\
		0                                & \leq2\left\|x\right\|
	\end{align*}
	\begin{math}
		\therefore
		\forall x\in X:
		\left\|x\right\|\geq
		0
	\end{math}.
\end{sol}

\begin{sol}{prob4}
	\leavevmode
	\begin{enumerate}[i)]
		\item $d\left(x,y\right)=0$ si y solo si $x=y$.


		      \begin{math}
			      d\left(x,y\right)=
			      0\iff
			      \left\|x-y\right\|=
			      0\iff
			      x-y=
			      0\iff
			      x=
			      y
		      \end{math}.

		\item

		      \begin{math}
			      d\left(x,y\right)=
			      d\left(y,x\right)\forall
			      x,y\in X
		      \end{math}.

		      \begin{math}
			      d\left(x,y\right)=
			      \left\|x-y\right\|=
			      \left\|-\left(y-x\right)\right\|=
			      \left|-1\right|\left\|y-x\right\|=
			      \left\|y-x\right\|=
			      d\left(y,x\right).
		      \end{math}


		\item

		      \begin{math}
			      d\left(x,y\right)\leq
			      d\left(x,z\right)+d\left(z,y\right)\forall
			      x,y,z\in X
		      \end{math}.

		      \begin{math}
			      d\left(x,y\right)=
			      \left\|x-y\right\|=
			      \left\|x-z+z-y\right\|\leq
			      \left\|x-z\right\|+\left\|z-y\right\|=
			      d\left(x,z\right)+d\left(z,y\right)
		      \end{math}.
	\end{enumerate}
\end{sol}

\begin{sol}{prob5}
	\begin{math}
		d\left(x,y\right)=
		\left\|x-y\right\|
	\end{math}
	es una métrica
	en $X$.
	Se tiene que
	\begin{math}
		d\left(x,y\right)\geq
		0\,\forall
		x,y\in X\implies
		\left\|x-y\right\|\geq
		0\,\forall x,y\in X
	\end{math}.
	Sea
	\begin{math}
		y=
		0\implies
		\left\|x\right\|\geq
		0\,\forall
		x\in X
	\end{math}.
\end{sol}

\begin{sol}{prob6}
	\begin{math}
		\left|
		\lambda_{k}
		\right|\geq
		0\,\forall
		k\in\mathbb{N}\implies
		\sum_{k=1}^{n}
		\left|
		\lambda_{k}
		\right|\geq
		0\implies
		\left\|
		x
		\right\|\geq
		0
	\end{math}.
\end{sol}

\begin{sol}{prob7}
	\leavevmode
	\begin{itemize}
		\item[$\left(\Rightarrow\right)$]

		      Se tiene que $\left\|x\right\|\geq 0\,\forall x\in X$.
		      Asumamos que $\left\|x\right\|=0\implies x=0$.
		      Contradicción.
		      $\therefore \left\|x\right\|>0$.

		\item[$\left(\Leftarrow\right)$]

		      Asumamos que
		      \begin{math}
			      x=
			      0\implies
			      \left\|x\right\|=
			      0
		      \end{math}.
		      Contradicción
		      $\therefore x\neq0$.
	\end{itemize}
\end{sol}

\begin{sol}{prob8}
	\begin{align*}
		\shortintertext{Sea $x\in\mathbb{R}^{n}$.}
		\varphi\left(x\right)\in X
		\left\|\varphi\left(x\right)\right\|\geq
		0 \\
		{\left\|x\right\|}_{\ast}\geq
		0.
	\end{align*}
\end{sol}

\begin{sol}{prob9}
	.
\end{sol}

\begin{sol}{prob10}
	.
\end{sol}

\begin{sol}{prob11}
	.
\end{sol}

\begin{sol}{prob12}
	.
\end{sol}

\begin{sol}{prob13}
	.
\end{sol}

\begin{sol}{prob14}
	.
\end{sol}

\begin{sol}{prob15}
	.
\end{sol}

\begin{sol}{prob16}
	.
\end{sol}

\begin{sol}{prob17}
	.
\end{sol}

\begin{sol}{prob18}
	.
\end{sol}

\begin{sol}{prob19}
	.
\end{sol}

\begin{sol}{prob20}
	.
\end{sol}
