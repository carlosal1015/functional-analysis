% arara: lualatex: {
% arara: --> shell: no,
% arara: --> draft: yes,
% arara: --> interaction: batchmode
% arara: --> }
% arara: lualatex: {
% arara: --> shell: no,
% arara: --> draft: no,
% arara: --> interaction: batchmode
% arara: --> }
% arara: clean: {
% arara: --> extensions:
% arara: --> ['aux','log','toc']      
% arara: --> }
\documentclass[a4paper]{scrbook}
\usepackage[spanish]{babel}
\spanishdatedel
\usepackage{mathtools}
\usepackage{amssymb}
\usepackage{amsthm}
\theoremstyle{definition}

\newtheorem*{theorem*}{Teorema}
\newtheorem{theorem}{Teorema}
\newtheorem*{definition*}{Definición}
\newtheorem{definition}{Definición}
\newtheorem*{proposition*}{Proposición}
\newtheorem{proposition}{Proposición}

\usepackage[shortlabels]{enumitem}

\renewcommand*{\proofname}{Prueba}

\titlehead{{\Large Universidad Nacional de Ingeniería
            \hfill 2024-1\\}
    Facultad de Ciencias\\
    Av. Túpac Amaru N$^{\circ}$ 210 Rímac}
\subject{Libro}
\title{Análisis funcional}
\subtitle{Ejercicios resueltos}
\author{Prof. Héctor Carlos Guimaray Huerta}
\date{\today}
% \publishers{Adviser Prof. John Doe}
\lowertitleback{Este libro de ejercicios fue digitado por Carlos Alonso Aznarán Laos con la ayuda de {\KOMAScript} y {\LaTeX}.}
% \uppertitleback{Self-mockery Publishers}
\dedication{Dedicado a lo(a)s matemático(a)s \\
    peruano(a)s del pasado y del futuro.}

\begin{document}


\maketitle
\tableofcontents

\chapter{Capítulo 1}
\section{Espacio vectorial normado}

\begin{definition*}
    Sea $X$ un espacio vectorial sobre $\mathbb{R}$ $\left(\mathbb{C}\right)$, $\left\|\phantom{\cdot}\right\|\colon X\to\mathbb{R}$ se llama norma real (compleja) en $X$ si:
    \begin{enumerate}[i)]
        \item

              \begin{math}
                  \left\|x\right\|=
                  0\implies
                  x=0
              \end{math}.
              Definida, separación.

        \item

              \begin{math}
                  \left\|\lambda x\right\|=
                  \left|\lambda\right|
                  \left\|x\right\|
                  \forall x\in X, \lambda\in\mathbb{R} \left(\mathbb{C}\right)
              \end{math}.
              Absolutamente Homogénea,
              Homogeneidad Positiva.

        \item

              \begin{math}
                  \left\|x+y\right\|\leq
                  \left\|x\right\|+
                  \left\|y\right\|
                  \forall x, y\in X
              \end{math}.
              Desigualdad Triangular, Subaditiva.
    \end{enumerate}
    $\left\|x\right\|$ se llama norma de $x$.
    $\left(X, \left\|\phantom{\cdot}\right\|\right)$ se llama espacio vectorial real (complejo) normado.
\end{definition*}

\begin{proposition*}
    Sea $X$ un espacio vectorial normado.
    Luego,
    \begin{math}
        x=0\implies
        \left\|x\right\|=0
    \end{math}
\end{proposition*}

\begin{proof}
    \begin{align*}
        \shortintertext{$x=0\implies$}
        \left\|x\right\| & =\left\|0\right\|                \\
                         & =\left\|0x\right\|               \\
                         & = \left|0\right|\left\|x\right\| \\
                         & = 0\left\|x\right\|              \\
                         & =0
    \end{align*}
\end{proof}

\begin{proposition*}
    Sea $X$ un espacio vectorial normado.
    Luego, $x\neq0\implies\left\|x\right\|\neq0$.
\end{proposition*}

\section{Subespacio vectorial normado}
\section{Isomorfismo en espacio vectorial normado}
\section{Distancia en espacio vectorial normado}
\section{Diámetro de conjunto en espacio vectorial normado}
\section{Bola en espacio vectorial normado}
\section{Esfera en espacio vectorial normado}
\section{Normas equivalentes en espacio vectorial normado}
\section{Conjunto abierto en espacio vectorial normado}
\section{Conjunto cerrado en espacio vectorial normado}
\section{Vecindad en espacio vectorial normado}
\section{Entorno en espacio vectorial normado}
\section{Topología fuerte en espacio vectorial normado}
\section{Punto interior en espacio vectorial normado}
\section{Punto exterior en espacio vectorial normado}
\section{Punto de frontera en espacio vectorial normado}
\section{Punto aislado en espacio vectorial normado}
\section{Punto de acumulación en espacio vectorial normado}
\section{Conjunto Bolzano-Weierstrass en espacio vectorial normado}
\section{Punto de clausura en espacio vectorial normado}
\section{Conjunto separado en espacio vectorial normado}
\section{Conjunto denso en espacio vectorial normado}
\section{Espacio vectorial normado separable}
\section{Lema de Riesz en espacio vectorial normado}
\section{Sucesión en espacio vectorial normado}
\section{Punto límite de sucesión en espacio vectorial normado}
\section{Límite de sucesión en espacio vectorial normado}
\section{Sucesión de Cauchy en espacio vectorial normado}
\section{Teorema de Bolzano-Weierstrass para conjunto infinito en espacio vectorial normado finito dimensional}
\section{Teorema de Bolzano-Weierstrass para sucesión en espacio vectorial normado finito dimensional}
\section{Base de Schauder en espacio vectorial normado}
\section{Base de Hilbert en espacio vectorial normado}
\section{Función en espacio vectorial normado}
\section{Isometría en espacio vectorial normado}
\section{Función contraída en espacio vectorial normado}
\section{Límite de función en espacio vectorial normado}
\section{Función continua en espacio vectorial normado}
\section{Función uniformemente continua en espacio vectorial normado}
\section{Función homeomorfa en espacio vectorial normado}
\section{Sucesión de funciones en espacio vectorial normado}
\section{Límite de sucesión de funciones en espacio vectorial normado}
\section{Serie de funciones en espacio vectorial normado}
\section{Conjunto acotado en espacio vectorial normado}
\section{Conjunto totalmente acotado en espacio vectorial normado}
\section{Conjunto completo en espacio vectorial normado}
\section{Función lineal continua en espacio vectorial normado}
\section{Función lineal acotada en espacio vectorial normado}
\section{Espacio de funciones lineales acotadas en espacio vectorial normado}
\section{Colección puntualmente acotado en espacio vectorial normado}
\section{Colección uniformemente acotado en espacio vectorial normado}
\section{Espacio dual topológico en espacio vectorial normado}
\section{Conjunto compacto en espacio vectorial normado}
\section{Conjunto conexo en espacio vectorial normado}
\section{Conjunto convexo en espacio vectorial normado}

\chapter{Capítulo 2}
\section{Espacio de Banach}

\chapter{Capítulo 3}
\section{Espacio producto interno}
\section{Ortogonalidad en espacio producto interno}
\section{Teorema de Pitágoras en espacio producto interno}
\section{Desigualdad de Bessel en espacio producto interno}

\chapter{Capítulo 4}
\section{Espacio de Hilbert}
\section{Límite de sucesión espacio de Hilbert}
\section{Teorema de Banach-Steinhaus en espacio de Hilbert}
\section{Teorema de Representación de Riesz en espacio de Hilbert}
\section{Límite débil de sucesión en espacio de Hilbert}
\section{Teorema de descomposición ortogonal en espacio de Hilbert}

\chapter{Capítulo 5}
\section{Teorema de extensión de Hahn-Banch en espacio vectorial}
\section{Principio de Acotación Uniforme en espacio de Banach}
\section{Teorema del gráfico cerrado en espacio de Banach}
\section{Teorema de la función abierta en espacio de Banach}

\end{document}